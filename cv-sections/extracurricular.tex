%----------------------------------------------------------------------------------------
%	SECTION TITLE
%----------------------------------------------------------------------------------------

\cvsection{Internal Projects}

%----------------------------------------------------------------------------------------
%	SECTION CONTENT
%----------------------------------------------------------------------------------------

\begin{cventries}

%------------------------------------------------

\cventry
{Realization of Easin Sales Server } % Affiliation/role
{ EasySoft-IN } % Organization/group
{ \href{https://bitbucket.org/account/user/easysoftin/projects/EAS}{BitBucket Project Link} }% Location
{Nov. 2018 - PRESENT} % Date(s)
{ % Description(s) of experience/contributions/knowledge
\begin{cvitems}
\item {The ESS is three part project, containing :}
\begin{itemize}
\item { A J2ee rest spring java server with a PostgreSQL Database behind called ESServer}
\item { An Angular 7 Server based Angular Material UI called EBILL }
\item { A phone App still to come}
\end{itemize}
\item {  ESServer : }
\begin{itemize}
\item{ The FJServer will save and manage invoices created by user from EBILL}
\item{ manage also clients , company ..}
\item{ It propose a Rest Api proteted by json Token mechanism.}
\end{itemize}
\item {EBILL :}
\begin{itemize}
\item {It's the Server Front END , A Web applications based on Anuglar 7 , and on Angular Matriel.}
\item { It will allow User to connect to ESServer}
\item { It will allow User to define invoices and manages saved invoices, export them in pdf ...}
\end{itemize}
\end{cvitems}
}

%------------------------------------------------

\cventry
{Realization of FJ Streaming Engine} % Affiliation/role
{ EasySoft-IN } % Organization/group
{ \href{https://bitbucket.org/account/user/easysoftin/projects/FJSE}{BitBucket Project Link} }% Location
{Jun. 2012 - Jan. 2017}  % Date(s)
{ % Description(s) of experience/contributions/knowledge
\begin{cvitems}
\item {The FJSE is three part project, containing :}
\begin{itemize}
\item { A J2ee rest java server  with a mongodb Database called FJServer}
\item { An AngularJs 5 Server UI called FJUI }
\item { A  html 5 Js dash player called FJPlayer}
\end{itemize}
\item {  FJServer : }
\begin{itemize}
\item{ The FJServer will upload video content ,  transcode it(using ffmpeg) to 5 bitrates h264 video ( from 360kb/s to 4500kb/s) and 3 bitrates aac audio( from 128kb/s to 450 kb/s , then dash and encrypt it using Bento4 framework for vod and Googlepackager for Live (packaging is done wi python scripts and CENC clear key  is used ). once done , the movie is available to FJSE client to be streamed as dash protected content . }
\item {this FJServer is used by manager user to connect using oauth2 protocol (login/password), than he can upload medias and define streams, slso define app or web site allowed to access this stream , once dashed streams will be availables to allowed app or web sites for streaming}
\end{itemize}
\item {FJUI :}
\begin{itemize}
\item {it the Server Angular 5 UI , which allow User  to connect to FJServer , define stream and upload movies or define live streams  , user also can define app/webSite that are allowed to use his streams ..
 streaming}
\item {The UI contains also a dashboard that presents statistics about streams regarding number of watches , clients IP ..}
\end{itemize}
\item { FJPLAYER :}
\begin{itemize}
\item {The FJPlayer Html 5 Js player based on Video Balise }
\item {The FJPlayer also is based On \href{http://dashif.org/reference/players/javascript/1.4.0/samples/dash-if-reference-player/}{Dashjs player} }
\item {Once we ask to play a media content on website , the player will ask the encryption key of the media presenting the web site AppID, then get the mpd and start streaming , decrypting  and playing..}
\end{itemize}
\end{cvitems}
}

%------------------------------------------------

\cventry
{Realization of an android application } % Affiliation/role
{Easy UPNP} % Organization/group
{ \href{https://play.google.com/store/apps/details?id=com.EasySoft.easyup}{Easy Upnp on Google Play Link} } % Location
{Nov. 2012} % Date(s)
{ % Description(s) of experience/contributions/knowledge
\begin{cvitems}
\item {The applicartion is an upnp control point wthat allow to discover and control upnp devices on a lan network.}
\item {This app was completly developpend with Java code.}
\item {This app also contains advertissment , it's done with google AdsMob..}
\item {This app is no more maintained}
\end{cvitems}
}

%------------------------------------------------


\end{cventries}