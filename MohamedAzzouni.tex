%%%%%%%%%%%%%%%%%%%%%%%%%%%%%%%%%%%%%%%%%
% Awesome Resume/CV
% XeLaTeX Template
% Version 1.1 (9/1/2016)
%
% This template has been downloaded from:
% http://www.LaTeXTemplates.com
%
% Original author:
% Claud D. Park (posquit0.bj@gmail.com) with modifications by 
% Vel (vel@latextemplates.com)
%
% License:
% CC BY-NC-SA 3.0 (http://creativecommons.org/licenses/by-nc-sa/3.0/)
%
% Important note:
% This template must be compiled with XeLaTeX, the below lines will ensure this
%!TEX TS-program = xelatex
%!TEX encoding = UTF-8 Unicode
%
%%%%%%%%%%%%%%%%%%%%%%%%%%%%%%%%%%%%%%%%%

%----------------------------------------------------------------------------------------
%	PACKAGES AND OTHER DOCUMENT CONFIGURATIONS
%----------------------------------------------------------------------------------------
\documentclass[11pt, a4paper]{awesome-cv} % A4 paper size by default, use 'letterpaper' for US letter
\geometry{left=2cm, top=1.5cm, right=2cm, bottom=2cm, footskip=.4cm} % Configure page margins with geometry
\fontdir[fonts/] % Specify the location of the included fonts

% Color for highlights
\colorlet{awesome}{awesome-red} 
% Default colors include: awesome-emerald, awesome-skyblue, awesome-red, awesome-pink, awesome-orange, awesome-nephritis, awesome-concrete, awesome-darknight
%\definecolor{awesome}{HTML}{F00A12} % Uncomment if you would like to specify your own color

% Colors for text - uncomment and modify
%\definecolor{darktext}{HTML}{424242}
%\definecolor{text}{HTML}{151515}
%\definecolor{graytext}{HTML}{1C1C1C}
%\definecolor{lighttext}{HTML}{414141}

\renewcommand{\acvHeaderSocialSep}{\quad\textbar\quad} % If you would like to change the social information separator from a pipe (|) to something else

%----------------------------------------------------------------------------------------
%	PERSONAL INFORMATION
%	Comment any of the lines below if they are not required
%----------------------------------------------------------------------------------------

\name{Mohamed}{Azzouni}
\address{8 rue Jesse Owens , Saint Denis}
\mobile{0033 (0) 699879354}
\email{mohamed.azzouni@gmail.com}
\homepage{www.easysoft-in.com}
\github{medazzo}
\linkedin{mohamedazouni}
%\skype{skypeid}
%\stackoverflow{SOid}{SOname}
%\twitter{@twit}
\position{Senior Software Engineer{\enskip\cdotp\enskip} Linux/Yocto - STB/DRM/OTT - Java/kotlin/Spring -  Javascript/Nodejs/Angular - Jenkins/DevOps} % Your expertise/fields

\makecvfooter{\today}{Mohamed Azzouni~~~·~~~CV }{\thepage} % Specify the letter footer with 3 arguments: (<left>, <center>, <right>), leave any of these blank if they are not needed

%----------------------------------------------------------------------------------------

\begin{document}
\makecvheader % Print the header

%----------------------------------------------------------------------------------------
%	CV/RESUME CONTENT
%	Each section is imported separately, open each file in turn to modify content
%----------------------------------------------------------------------------------------

\cvsection{Skills}
\begin{cvskills}

    \cvskill
    {Programming} % Category
    {C, C++, JAVA, KOTLIN, BASH, PYTHON, PERL, SHELL , Makefiles, HTML, PHP, XML, Javascript  }% Skills

    \cvskill
    {Tools \& CI} % Category
    {SVN, Git, Github, Gitlab, Bitbucket, Jenkins, JIRA, Confluence }% Skills

    \cvskill
    {Web \& OTT} % Category
    {nodejs, dashjs,reactjs, expressjs, angular, shaka player, Packet Video, GUPNP, Lib Platinum, REST }% Skills

    \cvskill
    {J2ee \& DB} % Category
    {Java Jersey Rest, J2ee, Spring, Rest, PostgresQl, Mongo db, Hbase} % Skills

    \cvskill
    {Embedded} % Category
    { Android: NDK/SDK/JNI, STB:Middleware \& Drivers, OpenEmbedded:Yocto \& sBitbacke}
    
    \cvskill
    {Networking} % Category
    { HDCP, MPEGDASH, HLS , VOD, OTT,Playready, UPNP, IPV4, TCPIP, Serial, ..}% Skills

    \cvskill
    {Languages} % Category
    {Arabic, French, English} % Skills

\end{cvskills}


\cvsection{Education}
\begin{cventries}

    \cventry
    {Engineering degre in Computer Science} % Degree
    {ENSI( National School of Computer Sciences of Tunis )} % Institution
    {Tunis, Tunisia} % Location
    {2004 - 2007} % Date(s)
    { % Description(s) bullet points
        speciality in open systems and networks
    }

\end{cventries}


\cvsection{Experience}
\begin{cventries}

    %------------------- ELEKTROBIT -----------------------------

 \cventry
    {Linux Yocto Expert} % Job title
    {WABTEC} % Organization
    {Hybrid:  Tours , France} % Location
    {Juin. 2023 - Present} % Date(s)
    { % Description(s) of tasks/responsibilities
        \begin{cvitems}
            \item {Migrate svn yocto repository  from svn to gitlab }
            \item {Setup CI using KAS and gitlob ppipelines for complete yocto configuration}
            \item {Migrate Yocto dunfell to yocto kirkstone for all Sama5 platforms}
            \item { \textbf{Environment :} SAMA5, SVN, Gitlab, Linux, Yocto , Device Tree, Uboot, yocto dunfell , yocto kirkstone, kernel, modules kernel}
        \end{cvitems}
    }

 \cventry
    {Linux Yocto Expert} % Job title
    {WABTEC} % Organization
    {Hybrid:  Tours , France} % Location
    {December. 2022 - December. 2023} % Date(s)
    { % Description(s) of tasks/responsibilities
        \begin{cvitems}
            \item {Build yocto Linux distrubution for HW with bitbus serial connection}
            \item {develop kernel module that manage a network interface and make the link with the serial bitbus connection}
            \item { \textbf{Environment :} SAMA5 ,SVN, Linux, Yocto , Device Tree, Uboot, yocto dunfell , yocto kirkstone, kernel, modules kernel}
        \end{cvitems}
    }

 \cventry
    {Safe Api DevOPS Developer} % Job title
    {ELEKTROBIT} % Organization
    {Carrières-sur-Seine , France} % Location
    {September. 2021 - Februray 2023} % Date(s)
    { % Description(s) of tasks/responsibilities
        \begin{cvitems}
            \item {In project Safe Linux for EB, I'm responsible of packages builds as well as documents Builds}
            \item {Setting up a Poc for CI Framework as a python package}
            \item {Setting a Cmake build system for documenrts builds}                        
            \item { \textbf{Environment :} Linux, Suse Open Build Service, GITHUB , Jenkins, Python, Cmake, docbook, docker}
        \end{cvitems}
    }


    \cventry
    {EBLinux-3 Architecte} % Job title
    {ELEKTROBIT} % Organization
    {Carrières-sur-Seine , France} % Location
    {September. 2020 - Februray 2023} % Date(s)
    { % Description(s) of tasks/responsibilities
        \begin{cvitems}
            \item {Architect: EBLinux-3 is a safe Linux Automotive distribution for EB, based on SUSE Open Build Service and Suse Kernel}
            \item {This new Version in no more based on yocto, here we deliver rpm packages using OBS}
            \item {Responsible for setting test environment:}
            \begin{itemize}
                \item adding needed repository , and packages 
                \item Setting up EBS packages definitions   
                \item setting up Building environement 
                \item setting up needed scripts to run qemu 
                \item setting up needed scripts to install image  on Hardware Targets IntelDenverton and makeavailable for all team
              \end{itemize}
            \item {Responsible for setting coverage tests.}
            \item { \textbf{Environment :} Linux, Suse Open Build Service, Intel Denverton , code coverage, obj coverage CI, Jenkins}
        \end{cvitems}
    }

    \cventry
    {EBLinux Architecte } % Job title
    {} % Organization
    {Carrières-sur-Seine , France} % Location
    {August. 2018 - August. 2020} % Date(s)
    { % Description(s) of tasks/responsibilities
        \begin{cvitems}
            \item {Architecte on EB-Linux: EB-Linux is an Automotive Linux distribution for EB, based on Yocto Open-embedded tools}
            \item {This Linux distribution will manage containers to run user applications }
            \item {This Linux distribution support RCAR H3/M3 and Intel on bare Metal or on A Hypervisor based on L4 (Fiasco) Microkernel}
            \item {My intervention :}
            \begin{itemize}
                \item Support on Requiremnts definitions and redaction
                \item Techniqual tickets creation on JIRA based on Functional or Client Requirements.
                \item Recipes implementations and maintaining related to network area
                \item Client Project task force support on bitbake and recipes hot fixes:
                \begin{itemize}
                    \item removing openssl
                    \item enabling SMACK Mandatory ACCESS Control Kernel 
                    \item ..
                  \end{itemize}
              \end{itemize}
            \item { \textbf{Environment :} Linux,Yocto, Renesas ,L4 Microkernel, CI, Jenkins, RobotFramework, LTP, BASH, C/C++}
        \end{cvitems}
    }

    \cventry
    {EBLinux CI Team Techniqual Lead} % Job title
    {} % Organization
    {Carrières-sur-Seine , France} % Location
    {Sep. 2018 - August. 2019} % Date(s)
    { % Description(s) of tasks/responsibilities
        \begin{cvitems}
            \item {Techniqual Responsible for Continous Integration Team: plan, support team and follow activity}
            \item {The team is made up of six engineers based in Croatia }
            \item {Team goals was to setup testing unit testing , integrations .. Merging verifications on nightly and weekly frequency}
            \item {Tests are setup on Jenkins Server's and robot frameworks tests for LTP tests on Yocto Framework}
            \item {Tests are run on Qemu and Hardware targets using tools and scripts created within the team and results are saved in Github}
            \item {CI is based on Github and Jenkins using Merging Trigger on Pull Requests}
            \item { \textbf{Environment :} Linux,Yocto, python, xml, CI, Jenkins, RobotFramework, BASH, C/C++}
        \end{cvitems}
    }

    %------------------- EXPWAY -----------------------------

    \cventry
    {Linux/Android Development Engineer } % Job title
    {EXPWAY} % Organization
    {PARIS , France} % Location
    {Jan. 2018 - Jui. 2018} % Date(s)
    { % Description(s) of tasks/responsibilities
        \begin{cvitems}
            \item {Development of Solution for Public Gateway EMBMS Live Services WIFI multicast}
            \item {LTE eMBMS (evolved Multimedia Broadcast Multicast Service) are used to drive live multicast service to end userphones.}
            \item { An LTE  Outdoor unit is connected to a Gateway, the soft to be developed will be deployed on PG to activate dash live services and forward multicast packet to WIFI.}
            \item {based on Expway Libarary a service will be developped and installed as daemon onBroadcom based public Gateway}
            \item { \textbf{Environment :} Modem AT-Command, Broadcom ,Gateway, WIFI, Multicast, dash, EXPWAY Middleware , redmine , C, C++, Android}
        \end{cvitems}
    }


    %------------------- SFR -----------------------------

    \cventry
    {Software DRM/HDCP Engineer } % Job title
    {SFR-Numericable} % Organization
    {Saint-Denis , France} % Location
    {Feb. 2016 - Dec. 2017} % Date(s)
    { % Description(s) of tasks/responsibilities
        \begin{cvitems}
            \item {Integration of Netflix nrd 4.3 within new 4K STB .}
            \item {Concept and develop a library HDCPmanager that provides interface to manage hdcp link in 1.X and 2.x protocole versions.}
            \item {Concept and develop a library DrmManager that provides interfaces to manager Playready 2.5 and eventually Widevine for SFR new Box used for VOD playing }
            \item {Concept and develop a library that will install provisioning Data on STB : On factory for SFR new Box ( data : Nagra, HDCP, DRM and SFR keys )}
            \item {Integration of STMicroelectronics PlayReady SDK in SFR Set Top Boxes products}
            \item {Developing a library interface of STM PlayReady SDK , to be used by SFR Middleware on Set Top Box .}
            \item { \textbf{Environment :} STB, STM, Broadcom , Middleware WYPLAY, MW SFR , C, C++, PlayReady 1.2/2.5,TEE, Gentoo, mercurial, Git}
            \item { \textbf{References :} Sbastien Keller(SFR), François Delavaud , Anouar Chelbia (SFR), Vishal Sharma ( Netflix)}
        \end{cvitems}
    }


    \cventry
    {Netflix integration } % Job title
    {}{}
    {Aug. 2014 - Jan.2016} % Date(s)
    { % Description(s) of tasks/responsibilities
        \begin{cvitems}
            \item {Integration of Netflix Partner Client Code in Middleware code on Set Top Box Based on 7105 Chip of STM.}
            \item {Porting some parts of Netflix client code (audio and video driver  and some Direct FB code ) to work on 7105 based Set Top Box}
            \item {Realization of automatic Netflix tests and certification }
            \item {Monitoring with Netflix partner via weekly conferences calls between team in US and France and daily coordination integration}
            \item { \textbf{Environment :}  STM, MW SFR , C, C++, PlayReady 1.2, DirectFb ,TEE, Gentoo, mercurial, Git}
            \item { \textbf{References :} François Delavaud , Anouar Chelbia (SFR), Vishal Sharma ( Netflix)}
        \end{cvitems}
    }
    %---------------- Bouygues --------------------------------    

    \cventry
    {Software integration \& developping Engineer} % Job title
    {Bouygues télécom} % Organization
    {Vélizy-villacoublay , France} % Location
    {Feb. 2014 - Jun. 2014} % Date(s)
    { % Description(s) of tasks/responsibilities
        \begin{cvitems}
            \item {Development of software module that use a cloud client ( based on PogoPug Fuse solution); So the user could connect , mount  and use his pogo cloud files on her STB}
            \item {Development and integration of Media Center new features on Middleware.}
            \item { \textbf{Environment :} STB, Middleware Sagem ,Product Bouygues Sensation Box ,  C, C++, Linux, UPNP, DLNA, POGOCLOUD Cloud, Fuse File system,  PacketVideo }
            \item { \textbf{References :} Sandrine Figard (Bouygues telecom), Majed Jabri (Bouygues telecom) }
        \end{cvitems}
    }

    \cventry
    {Architecte STB for PVR Sharing Service} % Job title
    {}{}
    {Jun. 2013  - Jan. 2014}
    { % Description(s) of tasks/responsibilities
        \begin{cvitems}
            \item {Writing of SSA  (Software Specification Architecture) of service STB PVR sharing (PVR: Personnel Video Recording)  : secured solution based on Protocol HLS  to play STB recorded movies on LAN and WAN devices ; records will be transcoded , segmented , encrypted and served to clients).}
            \item {Writing of STS (Software tests Specifications) of PVR Sharing service on STB part of Service.  }
            \item { \textbf{Environment :} STB, Middleware Sagem, LibPlatinum, C, C++, Linux, UPNP, HTTP, HTTPS,SSL }
            \item { \textbf{References :}  Nicolas Gaude (Bouygues telecom), Jabri Majed(Bouygues telecom) }
        \end{cvitems}
    }

    \cventry
    {Software Enginner} % Job title
    {}{}
    {Sep. 2011 - May. 2013}
    { % Description(s) of tasks/responsibilities
        \begin{cvitems}
            \item {Develop Web Services STB Export module that will export multiple STB features (Remote Control, Pairing, PVR, EPG ..) via HTTP Rest and UPNP. }
            \item {Develop STB DMR : module proposing DLNA DMR service UPNP  and HTTP Rest. }
            \item {Develop MediaPlayer : module STB, for mixed playlist media managements}
            \item {Develop Receiver Expo STB : module that serves as a asynchronous full duplex communication library between the Middleware and Web Services Export Binary}
            \item {Develop PfsProxy : module on multiple platforms (STB, Android and IOS) of for connection and communication with Bouygues platforms (EPG, PVR, OAuth, Smartvision .. ) with automatic authentication tokens management.}
            \item {Develop DMC : module on multiple platforms (STB, Android and IOS)  it’s a UPnP control point }
            \item {Development of a Native Middleware on Android (NDK / C / C ++ / JNI / JAVA and JAVASCRIPT) composed from  DMC, DMS and PFSPROXY :
                        JNI , JAVA and JavaScript Layers for each library ..}
            \item { \textbf{Environment :}  Middleware Sagem, STB,   C, C++, Java, JNI, JavaScript, Android, Linux, UPNP,DLNA, SOAP, REST,  LibPlatinum...}
            \item { \textbf{References :}  Nicolas Gaude ( Bouygues telecom ), Jabri Majed( Bouygues telecom ), Pascal Souveaux( Bouygues telecom ), Joël Motoumassy ( Bouygues telecom )}
        \end{cvitems}
    }

    %--------------------- NDS ---------------------------

    \cventry
    {Consulting Engineer} % Job title
    {NDS-CISCO} % Organization
    {Issy les Moulineaux, France} % Location
    {March 2010 - Aug. 2011} % Date(s)
    { % Description(s) of tasks/responsibilities
        \begin{cvitems}
            \item {Development and integration of a module within Middleware STB ; it’s a MediaCenter module that allow the sharing and control of medias over networks via Upnp/Dlna Protocol.}
            \item {DMS (Digital Media Server) solution is Twonky from PacketVideo company }
            \item {Development of a module in C for control and management of the Twonky server via RPC and for sharing FTA PVR and media content from USB drives connected to the STB}
            \item {Development and JAVA JNI  layer of the module to be exploited later by the UI in JAVA}
            \item { \textbf{Environment :}Middleware MHP de NDS-CISCO, languages C and JAVA, Shell, Makefiles, UPNP, PV Twonky API, PV Twonky Server. }
            \item { \textbf{References :} Hassen Taleb (NDS-CISCO) , Eric Delaunay (NDS-CISCO) }
        \end{cvitems}
    }

    %---------------------- Sagem Defense --------------------------

    \cventry
    {Software Engineer Development} % Job title
    {Sagem Défense} % Organization
    {Massy-Palaiseau, France} % Location
    {Dec. 2009 - Feb. 2010} % Date(s)
    { % Description(s) of tasks/responsibilities
        \begin{cvitems}
            \item {TMA : Third Maintenance Application of Network Server System on aircraft Airbus A380}
            \item { \textbf{Environment :} C, C++, Java, Perl, Python, SGBD MYSQL, Linux .. }
        \end{cvitems}
    }

    %----------------------- Eurogiciel -------------------------

    \cventry
    {Software Engineer} % Job title
    {Eurogiciel} % Organization
    {Tunis, Tunisia} % Location
    {May. 2009 - Nov. 2009} % Date(s)
    { % Description(s) of tasks/responsibilities
        \begin{cvitems}
            \item {Design and implementation of an application that manage a Switch via i2C in the NSS (Network System Server) for the Airbus A380}
            \item {The PSM, interfaces with LDAP, VPN, an I2C interface to control the Switch and a  Flight switches.}
            \item {Test are automatized in virtual environment using Perl scripting and shared memory C programs . }
            \item { \textbf{Environment :} C, tested and integrated under Linux FreeBsd 5.3. , perl ..}
        \end{cvitems}
    }

    %---------------------- STMicro --------------------------

    \cventry
    {Set Top Box Low Layer development Engineer,} % Job title
    {STMicroelectronics} % Organization
    {Tunis, Tunisia } % Location
    {Out. 2007 – Apr. 2009} % Date(s)
    { % Description(s) of tasks/responsibilities
        \begin{cvitems}
            \item {Engineer in STAPI (ST-API)  team : responsible for development, maintenance and support for VOUT (Video Output Stage) and HDMI (High Definition Multimedia Interface) drivers for STM socks.}
            \item {Bug fixing  and support  for various clients. }
            \item {Porting of drivers code on new chips and update documentation of drivers for new socks }
            \item { \textbf{Environment :} C, OS21, STLinux, Clear Case, socks: ST7100, ST7109, ST7200 et ST7105. }
        \end{cvitems}
    }

    \clearpage
\end{cventries}


\cvsection{Internal Projects EASYSOFT-IN}
\begin{cventries}

    \cventry
    {Open Source Spring Rest Generator } % Affiliation/role
    {sbr-generator: Open source} % Organization/group
    { \href{https://github.com/medazzo/sbr-generator}{sbr-generator on Github} }% Location
    {} % Date(s)
    { % Description(s) of experience/contributions/knowledge
        \begin{cvitems}
            \item {This is a a rest server based on Spring Boot generator , first version it was an npm binary and current (v2) it's a complete python package.}
            \item {This python package provide a tools that will generate a rest spring server with multiple features from a simple yaml config file :}
            \begin{itemize}
                \item {jwt token is ready to be used with authentication}
                \item {can also enable and generate java tests files for created code }
                \item {security can be enabled on role based restrictions}
                \item {a generated database ,connection and models , ready to e used}
                \item {dev and prod profiels with different database}
                \item {...}
            \end{itemize}
        \end{cvitems}
    }

    %------------------------------------------------

    \cventry
    {Fjplayer Mpeg Dash/Mp4 video player  } % Affiliation/role
    {fjplayer: Open source } % Organization/group
    { \href{hhttps://github.com/medazzo/fjplayer}{fjplayer on Github} } % Location
    {Nov. 2012} % Date(s)
    { % Description(s) of experience/contributions/knowledge
        \begin{cvitems}
            \item {The FJPlayer Html 5 Js player based on Video Balise }
            \item {The FJPlayer also is based On \href{http://dashif.org/reference/players/javascript/1.4.0/samples/dash-if-reference-player/}{Dashjs player} }
            \item {Once we ask to play a media content on website , the player will ask the encryption key of the media presenting the web site AppID, then get the mpd and start streaming , decrypting  and playing..}
        \end{cvitems}
    }

    %------------------------------------------------

    \cventry
    {Realization of FJ Streaming Engine} % Affiliation/role
    {EKIOSK : Private Source} % Organization/group
    { \href{https://bitbucket.org/easysoftin/workspace/projects/ECK}{BitBucket Project Link} }% Location
    {Jun. 2012 - Jan. 2017}  % Date(s)
    { % Description(s) of experience/contributions/knowledge
        \begin{cvitems}
            \item {The EKIOSK is a yocto based project used to realize kiosk like images deployed on ARM raspberry pie, composed by :}
            \begin{itemize}
                \item { meta-eck : yocto layer that define all needed recipes }
                \item { include browser layers to run web kiosk application }            
                \item { packages source code : ..}            
            \end{itemize}
        \end{cvitems}
    }

    %------------------------------------------------

    \cventry
    {Realization of Easin Sales Server } % Affiliation/role
    { Forja Streaming Engine: Private Source} % Organization/group
    { \href{https://bitbucket.org/easysoftin/workspace/projects/EAS}{BitBucket Project Link} }% Location
    {Nov. 2018 - PRESENT} % Date(s)
    { % Description(s) of experience/contributions/knowledge
        \begin{cvitems}
            \item {The ESS is three part project, containing :}
            \begin{itemize}
                \item { A J2ee rest spring java server with a PostgreSQL Database behind called ESServer}
                \item { An Angular 7 Server based Angular Material UI called EBILL }
                \item { A phone App still to come..}
            \end{itemize}
            \item {  ESServer : }
            \begin{itemize}
                \item{ The FJServer will save and manage invoices created by user from EBILL}
                \item{ manage also clients , company ..}
                \item{ It propose a Rest Api proteted by json Token mechanism.}
            \end{itemize}
            \item {EBILL :}
            \begin{itemize}
                \item {It's the Server Front END , A Web applications based on Anuglar 7 , and on Angular Matriel.}
                \item { It will allow User to connect to ESServer}
                \item { It will allow User to define invoices and manages saved invoices, export them in pdf ...}
            \end{itemize}
        \end{cvitems}
    }

    %------------------------------------------------

    \cventry
    {Realization of an android application } % Affiliation/role
    {Easy Upnp: Private Source} % Organization/group
    { \href{https://play.google.com/store/apps/details?id=com.easysoftin.easyup}{Google Play Link}, \href{https://bitbucket.org/easysoftin/easyup/src/master/}{Github Link} } % Location
    {Nov. 2012 -Nov 2021} % Date(s)
    { % Description(s) of experience/contributions/knowledge
        \begin{cvitems}
            \item {The application is an upnp control point that allow to discover and control upnp devices on a lan network.}
            \item {This app was completly developpend with kotlin code ( a old previous version was done using Java).}
            \item {This app also contains advertissment , it's done with google AdsMob..}
        \end{cvitems}
    }

\end{cventries}
\end{document}
